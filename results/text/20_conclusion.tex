
К промежуточному отчету по учебной практике имеются следующие результаты:

\begin{enumerate}
\item В качестве основы для парсера была выбрана библиотека Angstrom.
\begin{itemize}
\item Выбор сделан на основе сравнения производительности различных библиотек, репозиторий которого находится на \textsc{GitHub}\footnote{\url{https://github.com/lev-malets/pc-rd-cmp}}
\end{itemize}
\item Начата работа над синтаксическим анализатором \ReScript{}.
\end{enumerate}

\noindent
%Репозиторий со всем сопутствующим кодом:
%\\\quad

\section{Заключение}
Разработка легко поддерживаемых синтаксических анализаторов является давно решенной задачей в теории, но продолжает оставаться нерешенной на практике. Различные генераторы LR анализаторов продолжают требовать на вход грамматике, а возникающие конфликты при разборе требуют от программиста переписывания грамматики порою неестественным образом. 
Только средствами LR нельзя разбирать контекстно-зависимые языки, обход этих ограничений порою принимает ad hoc характер. 

Нисходящие анализаторы имеют другой недостатков, основным из которых являются леворекурсивные анализаторы. Статически детектировать такие случаи сложно, динамическое детектирование приводит к существенным накладным расходам. 

Парсер-комбинаторы почти решают проблему переиспользования и поддержки синтаксического анализатора в функциональных языках, но по-умолчанию дают не самый эффективный парсер. Мы возлагаем большие надежды на проекты типа~\cite{parsley2020}, которые позволяют сочетать выразительную мощь парсер-комбинаторов и статические проверки. 

В контексте данной работы метод оптимизации с помощью staging может показать интересные перспективы. Например, можно попробовать реализовать синтаксические анализаторы трех похожих языков (\OCaml{}, \ReasonML{} и \ReScript{}) в одной кодовой базе, а после получать оптимизированные анализаторы соответствующих языков статически указывая ограничения, с учетом которых стоит порождать оптимальный синтаксический анализатор.
