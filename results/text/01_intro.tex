В индустрии создания синтаксических анализаторов доминируют два подхода по их построению. Восходящие анализаторы семейства LR осуществляют разбор "снизу вверх", распознавая сначала маленькие участки входной цепочки, а затем объединяя их в большие. Нисходящие анализаторы вначале делают предположения об устройстве входной цепочки,  пробуют провести синтаксический анализ, а в случае неудачи возвращаются назад (наверх) и пробуют другую альтернативу. 


Оба подхода имеют свои преимущества и недостатки. Традиционно восходящие синтаксические анализаторы порождаются из специального входного файла, описывающего грамматику на некотором языке. Таким образом, ограничения входного языка грамматики, зачастую, накладывают ограничения на языки, которые может распознавать сгенерированный анализатор. Например, большинство анализаторов "снизу вверх" испытывают трудности с распознаванием языков, которые выходят из класса КС. Также, во время объединения маленьких участков в большие могут возникать неоднозначности (так называемые \textsc{shift-reduce} и \textsc{reduce-reduce} конфликты), и синтаксический анализатор вынужден угадывать правильное поведение.

Нисходящие анализаторы методом \emph{рекурсивного спуска} обычно пишутся непосредственно на языке программирования общего назначения и поэтому могут распознавать перечислимые языки любых классов сложности. Однако, из-за этого размер конечного синтаксического анализатора может отличаться в большую сторону, по сравнению с восходящими анализаторами.
Сложность практической реализации нисходящих анализаторов заключается в том, что неаккуратно написанные синтаксические анализаторы, хоть и не пользуются понятием грамматики непосредственно, могут не завершаться на некоторых входах по тем же причинам, по которым не завершаются леворекурсивные грамматики. При этом, так как анализатор описывается не на специальном предметно-ориентированном языке, возможности по статическому определению незавершаемости ограничены.

Однако, существуют промежуточный подход по написанию анализаторов, базирующийся на понятии комбинаторов синтаксических анализаторов (\emph{парсер-комбинаторов}~\cite{monparsing}). 
Он также осуществляет синтаксический анализ сверху вниз, что делает его родственным рекурсивному спуску. С его использованием анализатор описывается с помощью композиций парсер-комбинаторов, специальных функций, реализованных на языке общего назначения. Фиксированный интерфейс парсер-комбинаторов позволяет рассматривать их как предметно-ориентиро\-ван\-ный язык программирования, программы на котором (с некоторыми ограничениями) можно статически анализировать, в том числе на завершаемость. Так как парсер-комбинаторы по сути предоставляют высокоуровневый интерфейс по написанию синтаксических анализаторов, то получающиеся анализаторы будут выглядеть компактнее, чем их аналоги, записанные с помощью рекурсивного спуска.


В данной учебной практике планируется провести эксперимент по переписыванию синтаксического анализатора языка \ReScript{} с использованием подхода парсер-комбинаторов. В ходе него планируется подтвердить, что, по сравнению с рекурсивным спуском, парсер-комбина\-торы позволяют получить более компактный, но несколько более медленный синтаксический анализатор. Также планируется численно оценить во сколько раз полученный синтаксический анализатор уменьшился и замедлился. 
В качестве исходного синтаксического анализатора был выбран \ReScript{} так как он, во-первых, уже реализован методом рекурсивного спуска, и, во-вторых, реализован на функциональном языке программирования, который позволяет естественным образом использовать подход парсер-комбинаторов.

\textcolor{red}{Добавить пару-тройку ссылок на восходящий, нисходящий, rescript: dragonbook + cайт }



\begin{comment}
Многие синтаксические анализаторы реализованы методом рекурсивного спуска,
например, анализаторы в clang или компиляторах Kotlin или TypeScript.
Данный метод позволяет проще восстанавливаться от ошибок,
но является более объёмным и сложным в поддержке и улучшении кода.

Альтернативным методом реализации синтаксических анализаторов является метод монадических парсер-комбинаторов.
Его отличие от рекурсивного спуска --- большая декларативность,
которая приносит "читабельность" кода и легкость в изменении.

К минусам второго подхода, предположительно, можно отнести
более низкую производительность, а также сложность в восстановлении от ошибок.

Для уточнения этой гипотезы предполагается переписать синтакси\-чес\-кий анализатор ReScript,
используя подход парсер-комбинаторов, и сравнить его производительность с исходным анализатором.
\end{comment}
